\begin{tikzpicture}[node distance = 2cm]
    \definecolor{myblued}{RGB}{0,114,189}
    \definecolor{myred}{RGB}{217,83,25}
    \definecolor{myyellow}{RGB}{237,137,32}
    \definecolor{mypurple}{RGB}{126,47,142}
    \definecolor{myblues}{RGB}{77,190,238}
    \definecolor{mygreen}{RGB}{32,134,48}
      \pgfplotsset{
        label style = {font=\fontsize{9pt}{7.2}\selectfont},
        tick label style = {font=\fontsize{7pt}{7.2}\selectfont}
      }
    
    \small  % 字体大小
    \tikzstyle{format}=[rectangle,draw,thin,fill=white]  % 定义语句块的颜色,形状和边
    % rectangle:矩形,可加圆角(rounded corners,逗号跟在形状后面即可)
    % trapezium:平行四边形
    % diamond:菱形
    \tikzstyle{test}=[diamond,aspect=2,draw,thin]  % 定义条件块的形状,颜色
    \tikzstyle{point}=[coordinate,on grid,]  % 像素点,用于连接转移线

    % 定义note
    \node[format](asm){在记事本事先编好程序,并修改后缀名为\texttt{.asm}};
    \node[format,below of=asm,node distance=10mm](masm){使用命令\texttt{masm yourfile.asm}生成目标文件};
    \node[test,below of=masm,node distance=20mm](if){0 wornings, 0 errors?};
    \node[format,below of=if,node distance=20mm](link){使用命令\texttt{link yourfile}进行连接操作};
    \node[format,below of=link,node distance=10mm](execute){执行\texttt{youfile.exe}文件};
    % 开始画线
    \draw[->](asm)--(masm);
    \draw[->](masm)--(if);
    \draw[->](if)--node[left]{Yes}(link);
    \draw[->](link)--(execute);
    \draw[->](if.west) -+ (-5,-3) -+ node[left]{No}(-5,0) -- (asm.west);
    % \draw[-](point2) to [in=-90,out=90];
    % \draw[->](point2)--(asm.west);
\end{tikzpicture}
    