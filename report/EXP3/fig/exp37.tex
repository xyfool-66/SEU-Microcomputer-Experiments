\begin{tikzpicture}[node distance = 2cm]
    \definecolor{myblued}{RGB}{0,114,189}
    \definecolor{myred}{RGB}{217,83,25}
    \definecolor{myyellow}{RGB}{237,137,32}
    \definecolor{mypurple}{RGB}{126,47,142}
    \definecolor{myblues}{RGB}{77,190,238}
    \definecolor{mygreen}{RGB}{32,134,48}
      \pgfplotsset{
        label style = {font=\fontsize{9pt}{7.2}\selectfont},
        tick label style = {font=\fontsize{7pt}{7.2}\selectfont}
      }
    
    \small  % 字体大小
    \tikzstyle{format}=[rectangle,draw,thin,fill=white]  % 定义语句块的颜色,形状和边
    % rectangle:矩形,可加圆角(rounded corners,逗号跟在形状后面即可)
    % trapezium:平行四边形
    % diamond:菱形
    \tikzstyle{test}=[diamond,aspect=2,draw,thin]  % 定义条件块的形状,颜色
    \tikzstyle{point}=[coordinate,on grid,]  % 像素点,用于连接转移线

    % 定义note
    \node[format](init){初始化数据段,将数据传入\texttt{DS}};
    \node[format,below of=init,node distance=10mm](0){设置结果数组位置\texttt{[DI]}、被乘数数组的位置、循环次数,\textcolor{red}{结果存放指针\texttt{[DI]}直接指向末尾后一个字节,把\$放好}};
    \node[format,below of=0,node distance=10mm](1){倒退一字节,从最后一字节开始放结果};
    \node[format,below of=1,node distance=10mm](2){把此时指针所指的数字搬到\texttt{AL}};
    \node[format,below of=2,node distance=10mm](3){被乘数指针\texttt{+1}};
    \node[format,below of=3,node distance=10mm](4){进行相乘操作};
    \node[format,below of=4,node distance=10mm](5){非压缩\texttt{BCD}码的调整乘法};
    \node[format,below of=5,node distance=10mm](6){将[DI]中算出来的上一位的进位加上};
    \node[format,below of=6,node distance=10mm](7){非压缩\texttt{BCD}码的调整加法};
    \node[format,below of=7,node distance=10mm](8){调整好的正确的低位给了\texttt{[DI]},\textcolor{red}{并通过\texttt{+30}调整成\texttt{ASCII}码}};
    \node[format,below of=8,node distance=10mm](9){结果指针\textcolor{red}{\texttt{-1}}};
    \node[format,below of=9,node distance=10mm](10){把调整好的正确的进位给了\texttt{[DI]}};
    \node[test,below of=10,node distance=20mm](11){是否达到循环次数?};
    \node[format,below of=11,node distance=20mm](12){单独把最高位调整成\texttt{ASCII}码};
    \node[format,below of=12,node distance=10mm](13){设置打印开始位置,并打印};
    \node[format,below of=13,node distance=10mm](end){结束,返回\texttt{DOS}};
    % 开始画线
    \draw[->](init)--(0);
    \draw[->](0)--(1);
    \draw[->](1)--(2);
    \draw[->](2)--(3);
    \draw[->](3)--(4);
    \draw[->](4)--(5);
    \draw[->](5)--(6);
    \draw[->](6)--(7);
    \draw[->](7)--(8);
    \draw[->](8)--(9);
    \draw[->](9)--(10);
    \draw[->](10)--(11);
    \draw[->](11)--node[left]{Yes}(12);
    \draw[->](12)--(13);
    \draw[->](13)--(end);
    \draw[->](11.west) -+ (-6,-13) -+ node[left]{No}(-6,-3) -- (2.west);
\end{tikzpicture}
